\documentclass[../main.tex]{subfiles}

\begin{document}
The rise of the Internet of Things and affordable cloud infrastructure has led to a connected world of devices and services working together to achieve better accessibility and connectivity in real-time. While these devices are resource-constrained, there is more focus on performance and availability rather on security. Also, accessibility being important, leads to various services open to the internet thereby creating a huge risk insecurity of the system. Leveraging these known vulnerabilities, proactive security strategies like Honeypots are developed and deployed with these flaws to attract exploits from attackers thereby making them vulnerable by design. Honeypots are classified into three types based on their interaction, service simulation into low, medium and high interaction honeypots. As the name suggests, low interaction honeypots are focused on emulating the basic protocol services and do not focus on high interaction. Low interaction honeypots are quick to set up, highly usable and provide basic logging and analysis features. These honeypots are developed to be deployed in resource constraint environments and perform event logging. Low interaction honeypots are also not security-critical as any damage caused by them as a result of exploitation are minimal 

In addition to various security measures adopted by the administrators, Honeypots form a good defensive tool as decoy systems, to identify vulnerabilities and detect attacks. They act as a proactive approach to identify malicious sources for attacks and also give an understanding of the vulnerable areas of the system. All communication to the honeypot is considered hostile since legitimate users have no obligation to access a honeypot. The use of honeypots as a detection tool date back to over a decade with honeypot projects being open and deployed since the year 2000. Protocols like Telnet and SSH have been widely used even today to establish communication with remote systems. Similarly, protocols like HTTP, FTP, SMTP, SNMP, SMB have been used as base network communication protocols for essential services. There are honeypots developed to emulate all the stated essential protocols. The Honeynet Project \cite{Honeynet} is focused on the development of honeypots for well-known protocols and environments over the years. Administrators have been using these honeypot projects in their DMZ environments to detect the attacks on these protocols over the years. However, it has been evident that these honeypots could be detected and identified. This leads to an argument about the use of outdated honeypots \cite{counting} and their contribution towards attack detection once they are identified.

Honeypot detection is important to notify the developers of these projects of the maintenance. Soon after these projects are released as open-source, very little attention is given to updating or patching. While bad maintenance of any software is a huge security risk, there is also a risk of these projects being deployed more over the world. Once the identity of the honeypots are revealed, their purpose and productivity is degraded. The attackers learn about the honeypot and stay away from engaging any interaction with these systems. Also, honeypots being vulnerable systems by design, form easy entry points for APT exploits from attackers. The main merit of a honeypot is achieved by its ability to remain undetected while continuing interaction with the attacker. It is very important to make sure that this feature is tested by the developers by assuming the role of attackers. Hence, the detection of honeypots provides a good approach to the efficiency of a honeypot. This is very critical especially in the case of low interaction honeypots because of the limited interaction simulations and behavior. There is very less consideration of testing these honeypots before release and deployment. Thus it is crucial and critical for honeypot developers and users to understand these detection mechanisms to make their honeypots more productive and efficient in their deployed environments. 


Internet-wide scanning has now been convenient and popular because of scanning platforms like Shodan and Censys. These platforms provide search based on region, ports, IP addresses, and domains to check the services vulnerable and open to the internet. Tools like Nmap \cite{NMap} and Xprobe2 provide good scanning abilities to scan the end systems. Considering internet-wide scanning as a first step and using the results, it is possible to apply various approaches to determine if the system is a honeypot. We summarize the detection techniques published so far and classify these approaches into Active and Passive approaches of honeypot detection. Further, we use the classification to propose a test suite for honeypots and evaluate various open-source honeypots. However, we also emphasize the fact that all low interaction honeypots can be detected given enough time for replay and analysis. Our primary intention is to classify methods to fingerprinting honeypots based on their interaction overhead. With this paper we contribute with the following:

 \begin{itemize}
    \item Provide an overview of Honeypot Fingerprinting techniques
    \item Classify the fingerprinting techniques based on minimal and advanced interaction. We also propose the idea of active and passive approaches to fingerprinting honeypots
    \item We use the classification techniques proposed to evaluate against popular open-source honeypots and provide remedies to defend against these approaches
 \end{itemize}
\end{document}