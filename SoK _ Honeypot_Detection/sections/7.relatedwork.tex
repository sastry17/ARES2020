\documentclass[../main.tex]{subfiles}
\begin{document}
Joni et al. \cite{Joni} provide a  summary of research on anti-honeypot methods and propose a taxonomy, based on detection vectors. They classify the detection vectors based on temporal, operational, hardware and environment of the target system. The authors provide informative methods used by malware botnets for detecting honeypots. However, there is no evaluation of the detection vectors in the proposed taxonomy. The future work proposes development of a honeypot system with dual interfaces support. The dual interfaces enable a user and software use the primary interface as a normal system and the secondary interface as a fake system. The secondary interface acts as a honeypot and any interaction raises an alarm. The authors also propose self aware and dynamic honeypots that adapt and recover from any recognised fingerprinting methods. The proposed taxonomy attempts to consider attacks from an automated botnet malware and does not consider a human driven attack vector. In 2004, Krawetz et al. \cite{krawetz2004anti} proposed anti-honeypot technology that identifies potential honeypots that capture mail spammers and web servers. The technology aims at using fingerprinting methods that exploit the limited simulation of systems. The author warns  that there can be more effective honeypot detection methods evolving in the future that are more accurate. The author provides no taxonomy of these methods. 
\end{document}