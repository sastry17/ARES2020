\documentclass[../main.tex]{subfiles}

\begin{document}
Fingerprinting methods are basically classified into active and passive based on interaction with the target system. Active fingerprinting involves creation of specific probes and using them to query the the target system to collect as much data possible. On the contrary, passive fingerprinting  make use of available meta data to determine information about the target. Passive fingerprinting limit communication with the target system as much possible. Remote attackers tend to use active fingerprinting methods more because passive approaches do not lead to an inference. Passive fingerprinting techniques depend on fingerprint databases to infer their findings. Fingerprint databases contain information about distinct responses by systems on interaction.
   
   We classify the detection methods discussed in sections before into minimal and maximal interaction methods. The classification is based on the communication levels required by the methods to obtain qualified information to fingerprint the target system. Basically, honeypot networks block communication with the attackers after a connection attempt is detected. This limits the detection and further exploitation of the honeypot from a probe network. Also, we believe that minimum interaction methods are effective with honeypots fingerprinting, to limit logging information of the attacker and to avoid being blocked.  

 \subsection{Low Effort Fingerprinting Techniques}
 Minimal Interaction methods aim at detection of honeypots with minimal or no direct interaction with the target system. They use meta data based techniques to fingerprint honeypots. It is necessary to correlate the information obtained from these methods to achieve conclusive results on fingerprinting. This reconnaissance technique is complex because it requires substantiate time from the attacker to collect and integrate the meta data to infer the existence of a honeypot. The advantage of using minimal interaction methods for fingerprinting are that the attacker information remains hidden and no backtracking is possible. These techniques do not require deep knowledge in fingerprinting. The techniques focus on gathering meta data by means of social engineering methods. 
 
 
 \subsection{High Effort Fingerprinting Techniques}
Maximal interaction techniques employ probe based techniques to interact with the target system to obtain specific operational information from the target system. They leave a massive footprint on the target machine during the probing process. This footprint enables honeypot administrators to trace the attackers or blacklist the IP addresses of attack sources. Though maximal interaction techniques provide better data for detection accuracy, they are very risky.
\end{document}