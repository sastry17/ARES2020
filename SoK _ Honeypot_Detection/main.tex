%%%%%%%%%%%%%%%%%%%%%%%%%%%%%%%%%%%%%%%%%%%%%%%%%%%%%%%%%%%%%%%%%%%%%%%%%%%%%%%%
%2345678901234567890123456789012345678901234567890123456789012345678901234567890
%        1         2         3         4         5         6         7         8

\documentclass[letterpaper, 10 pt, conference]{ieeeconf}  % Comment this line out
                                                          % if you need a4paper
%\documentclass[a4paper, 10pt, conference]{ieeeconf}      % Use this line for a4
                                                          % paper

\IEEEoverridecommandlockouts                              % This command is only
                                                          % needed if you want to
                                                          % use the \thanks command
\overrideIEEEmargins
% See the \addtolength command later in the file to balance the column lengths
% on the last page of the document

\usepackage[utf8]{inputenc}
\usepackage[T1]{fontenc}
\usepackage{float}

% The following packages can be found on http:\\www.ctan.org
%\usepackage{graphics} % for pdf, bitmapped graphics files
%\usepackage{epsfig} % for postscript graphics files
%\usepackage{mathptmx} % assumes new font selection scheme installed
%\usepackage{mathptmx} % assumes new font selection scheme installed
%\usepackage{amsmath} % assumes amsmath package installed
%\usepackage{amssymb}  % assumes amsmath package installed

\title{\LARGE \bf
SoK: Classifying Honeypot Fingerprinting Techniques
}

%%%% ForDouble Blind Version
%\author{Shreyas Srinivasa$^{1}$, Emmanouil Vasilomanolakis$^{2}$, Jens Myrup Pedersen$^{3}$% <-this % stops a space
%\thanks{*This work was not supported by any organization}% <-this % stops a space
%\thanks{$^{1}$S. Srinivasa - PhD Fellow at Department of Electronic Systems,
%        Aalborg University, Copenhagen, Denmark
%        {\tt\small shsr@es.aau.dk}}%
%\thanks{$^{2,3}$E. Vasilomanolakis, J. Pedersen - Asst. Professor at                          Department of Electronic Systems, Aalborg University,                         Copenhagen, Denmark
%        {\tt\small emv@es.aau.dk, jens@es.aau.dk}}%
%}


\begin{document}



\maketitle
\thispagestyle{empty}
\pagestyle{empty}


%%%%%%%%%%%%%%%%%%%%%%%%%%%%%%%%%%%%%%%%%%%%%%%%%%%%%%%%%%%%%%%%%%%%%%%%%%%%%%%%
\begin{abstract}
In defensive security, Honeypots play an important role to detect attacks by simulating the services of the target system. Over the years, many honeypots have been proposed, developed and deployed over the internet simulating essential protocols and services like HTTP, SSH, Telnet, SMB, SMTP, FTP and also industrial protocols like MODBUS and BARCnet. Survey statistics provided by F-Secure (rfr: F-secure white paper), show that the attack landscape caught by honeypots hosted by F-Secure, have tripled from 2018 with major targeted protocols being Telnet, SMB, SSH and  MSSQL. The rise in attacks can be in view of increase in smart IoT devices and also the accessibility of Cloud infrastructure. Further, resource constraints on IoT devices lead to poor security diligence to achieve performance and availability. Honeypots have been a reliable choice for proactive defense by  network administrators to detect and analyze attacks. Open-source low-interaction honeypots like Kippo, Cowrie, Glastopf, Dionaea and Conpot are the most widely deployed Honeypots (rfr: bitter harvest). As effective as  honeypots can be leveraged to detect attacks, they are also vulnerable to be exposed as decoys thereby reducing their productivity. The identity of honeypots and its ability to remain undetected is a valuable parameter towards its purpose. Studies have shown that these honeypots are easily detectable and can be fingerprint by various mechanisms. Probing, response analysis and  location/domain filtering are some methods through which the identity of honeypots can be revealed. In addition, machine learning based methods have inferred to increase the predictability and reduce false positives in detection. This paper provides an overview and a new basis for classification of different low interaction honeypot fingerprinting mechanisms . 
\end{abstract}


%%%%%%%%%%%%%%%%%%%%%%%%%%%%%%%%%%%%%%%%%%%%%%%%%%%%%%%%%%%%%%%%%%%%%%%%%%%%%%%%
\section{INTRODUCTION}

The rise of Internet of Things and affordable cloud infrastructure has led to a connected world of devices and services working together to achieve better accessibility and connectivity in real time. While these devices are resource constrained, there is more focus on performance and availability rather on security. Also, accessibility being important, leads to various services open to the internet thereby creating a huge risk in security of the system. Leveraging these known vulnerabilities, proactive security strategies like Honeypots are developed and deployed with these flaws to attract exploits from attackers thereby making them vulnerable by design.Honeypots are classified into three types based on their interaction, service simulation into low, medium and high interaction honeypots. As the name suggests, low interaction honeypots are focused on emulating the basic protocol services and do not focus on high interaction. Low interaction honeypots are quick to setup, highly usable and provide basic logging and analysis features. These honeypots are developed  to be deployed in resource constraint environments and basically perform event logging. Low interaction honeypots are also not security critical as any damage caused by them as a result of exploitation are minimal 

In addition to various security measures adopted by the administrators, Honeypots form a good defensive tool as decoy systems, to identify vulnerabilities and detect attacks. They act as a proactive approach to identify malicious sources for attacks and also give an understanding of the vulnerable areas of the system. All communication to the honeypot is considered hostile, since legitimate users have no obligation to access a honeypot. The use of honeypots as a detection tool date back to over a decade with honeypot projects being open and deployed since the year 2000. Protocols like Telnet and SSH have been widely used even today to establish communication with remote systems. Similarly protocols like HTTP, FTP, SMTP, SNMP, SMB have been the used as base network communication protocols for essential services. There are honeypots developed to emulate all the stated essential protocols. Administrators have been using these honeypot projects in their DMZ environments to detect the attacks on these protocols over years. However, it has been evident that these honeypots could be detected and identified. This leads to an argument about the use of outdated honeypots and their contribution towards attack detection once they are identified.

Honeypot detection is important to notify the developers of these projects of the maintenance. Soon after these projects are released as open source, very little attention is given to updating or patching. While, bad maintenance of any software is a huge security risk, there is also a risk of these projects being deployed more over the world. Once the identity of the honeypots are revealed, their purpose and productivity is degraded. The attackers learn about the honeypot and stay away from engaging any interaction with these systems. Also, honeypots being vulnerable systems by design, form easy entry points for APT exploits from attackers. The main merit of a honeypot is achieved by its ability to remain undetected while continuing an interaction with the attacker. It is very important to make sure that this feature is tested by the developers by assuming the role of attackers. Hence, the detection of honeypots provides a good approach to efficiency of a honeypot. This is very critical specially in the case of low interaction honeypots because of the limited interaction simulations and behavior. There is very less consideration of testing these honeypots before release and deployment. Thus it is crucial and critical for honeypot developers and users to understand these detection mechanisms to make their honeypots more productive and efficient in their deployed environments . 


Internet wide scanning has now been convenient and popular because of scanning platforms like Shodan and Censys. These platforms provide search based on region, ports, IP addresses and domains to check the services vulnerable and  open to the internet. Tools like Nmap and Xprobe2 provide good scanning abilities to scan the end systems. Considering internet wide scanning as a first step and using the results, it is possible to apply various approaches to determine if the system is a honeypot. We summarize the detection techniques published so far and classify these approaches into Active and Passive approaches of honeypot detection. Further, we use the classification to propose a test suite for honeypots and evaluate various open source honeypots. However, we also emphasize the fact that all low interaction honeypots can be detected given enough time for replay and analysis. Our primary intention is to classify methods to fingerprinting honeypots based on their interaction overhead. With this paper we contribute with the following:

 \begin{itemize}
    \item Provide an overview about Honeypot Fingerprinting techniques and its importance in determining the productivity of low interaction honeypots 
    \item Classify the fingerprinting techniques on the basis of minimal and advanced interaction. We also propose the idea of active and passive approaches to fingerprinting
    \item We use the classification techniques proposed to evaluate against popular open source honeypots and provide remedies to defend against active and passive approaches 
 \end{itemize}




\section{Background}
Honeypots are one of the strategies employed in defensive security to detect attacks from adversaries. They are designed and developed with an intention to mimic the real protocols and provide information about the attack to the administrators. Based on the interaction and service emulation, honeypots are classified into low, medium and high interaction honeypots. Security administrators often feel that low interaction honeypots are feasible to be deployed in their environment as they are not resource intensive and are less risky. Low interaction honeypots are limited in protocol emulations and hence are also easy targets for fingerprinting attacks. 

Fingerprinting techniques are used to identify systems by analysis on the behavioral information retrieved by probing.   
It is normally the first step employed by attackers to gain more insight about the end system before exploiting them. This technique has proved to be effective and there are various fingerprinting tools available that can accurately determine the operating system, kernel versions, protocol versions, device name and other attributes of the end system. However, determining these attribute information may involve multiple scans and intermittent probing. The ideal outcome of device fingerprinting infers the device and its attributes, based on which the attacker can orchestrate exploits. The primary focus of a honeypot lies in its ability to stay undetected. Attackers use fingerprinting as an initial setup to determine the validity of the end device before investing their resources on exploitation. If the identity of honeypot is revealed during the initial fingerprinting techniques, there productivity is degraded. This happens due to two reasons. Firstly, the attacker decides to broadcast his findings about the existence of the honeypot which alerts other attackers when they are looking for targets. Second, there is a derived fingerprinting technique for the target honeypot, which can be included in the primary scan of popular fingerprinting tools. p0f, Nmap, Uniscan, Metasploit, Nikto and cisco-torch are some of the open-source fingerprinting tools which are capable of detecting honeypots like Kippo, Cowrie and Glastopf. 

Identifying deviations in the behavior of a system and inferring fingerprints has been an active research topic. Early works [brumley D, comparetti P] involved theoretical approach based on binary analysis. All attacks towards honeypots are recorded or logged for further analysis. This process is an expensive operation considering the writes and file operations thereby creating delay in the response times of the honeypot. Further, honeypots utilize the virtual environments to save resources in their environment. Based on these ideas, Holz et al [detecting honeypots] proposed that honeypots could be detected due to increase in execution time of the commands from the attackers because of logging and sandboxing.Fu et al. [FU, X] also, suggested that the delay in execution and the response due to the virtualized network layer leveraged by honeypots could be a factor to infer the presence of the honeypot. At the Black Hat conference 2015, Cymmetria research [cite Black Hat 2015] proposed various factors by which honeypots could be fingerprinted and recommended new design strategies and development practices to overcome the detection. Alexander et al. [cite Bitter Harvest] provide a systematic fingerprinting approach for low-interaction honeypots. They leverage the faults of poorly maintained libraries referenced in the emulation of protocols in popular honeypots. The libraries used in the development of protocols in honeypots were not intended to reproduce the actual behavior of the protocol itself but used for the ease of development. Th authors present a fingerprinting approach by development of probes that trigger the response at the transport layer and were successful in identifying up to 7605 honeypots online. Their research is limited to SSH, Telnet and HTTP honeypots. Internet scanning tools like Shodan have also released Honeypot identification methods called Honeyscore [cite Shodan honeyscore].The tool accepts an IP address or a URL as an input and provides result if the end system is a honeypot. The tool is stated to be still in development but is capable of recognizing popular honeypots on the internet. Feng et al. [cite ICS detection with ML approach] propose a machine learning model to collect and classify the response data received from Industrial Control Systems open on the internet. The approach relies on the flawed implementation of industrial control protocols and the deployment configurations of the honeypots. 


\begin{table*}[ht!]
\centering
 \begin{tabular}{||c c c c c c c||} 
 \hline
 SSH & Telnet & HTTP & SMB & Database & ICS & IoT\\ [0.5ex] 
 \hline
 Kippo & Telnet-IOT-Honeypot & Dionaea & HoneySMB & mysql-honeypotd & Conpot & HoneyThing\\ 
 
 HosTaGe &  HosTaGe &  HosTaGe &  HosTaGe &  HosTaGe &  HosTaGe &  HosTaGe  \\
 
 Cowrie & Cowrie & Glastopf &  & pghoney & GasPot & Kako\\

 Blacknet & TPwd & Conpot &  & MongoDB-HoneyProxy & GridPot & IotPot\\
 
 Kojoney2 & MTPot & Nodepot & & & & \\
 
 MockSSH & Hontel &  &  & & & \\ [1ex] 
 \hline
\end{tabular}
\label{table:1}\caption{Honeypots}
\end{table*}








\section{Fingerprinting Techniques}

Based on previous work, we now present various fingerprinting techniques for honeypots. These techniques could help in creation of custom scripts 




 \section{Classifying Fingerprinting Techniques}
   


 \section{Evaluation of Honeypots based on Classification}
 
 
 
 \section {Countermeasures}
 
 
 
 \section {Related Work}
 
 
 
\section{CONCLUSIONS}


\addtolength{\textheight}{-12cm}   % This command serves to balance the column lengths
                                  % on the last page of the document manually. It shortens
                                  % the textheight of the last page by a suitable amount.
                                  % This command does not take effect until the next page
                                  % so it should come on the page before the last. Make
                                  % sure that you do not shorten the textheight too much.

%%%%%%%%%%%%%%%%%%%%%%%%%%%%%%%%%%%%%%%%%%%%%%%%%%%%%%%%%%%%%%%%%%%%%%%%%%%%%%%%



%%%%%%%%%%%%%%%%%%%%%%%%%%%%%%%%%%%%%%%%%%%%%%%%%%%%%%%%%%%%%%%%%%%%%%%%%%%%%%%%



%%%%%%%%%%%%%%%%%%%%%%%%%%%%%%%%%%%%%%%%%%%%%%%%%%%%%%%%%%%%%%%%%%%%%%%%%%%%%%%%

\section*{ACKNOWLEDGMENT}




%%%%%%%%%%%%%%%%%%%%%%%%%%%%%%%%%%%%%%%%%%%%%%%%%%%%%%%%%%%%%%%%%%%%%%%%%%%%%%%%

References are important to the reader; therefore, each citation must be complete and correct. If at all possible, references should be commonly available publications.

\begin{thebibliography}{99}

\bibitem{c1} G. O. Young, ``Synthetic structure of industrial plastics (Book style with paper title and editor),'' 	in Plastics, 2nd ed. vol. 3, J. Peters, Ed.  New York: McGraw-Hill, 1964, pp. 15--64.
\bibitem{c2} W.-K. Chen, Linear Networks and Systems (Book style).	Belmont, CA: Wadsworth, 1993, pp. 123--135.
\bibitem{c3} H. Poor, An Introduction to Signal Detection and Estimation.   New York: Springer-Verlag, 1985, ch. 4.
\bibitem{c4} B. Smith, ``An approach to graphs of linear forms (Unpublished work style),'' unpublished.
\bibitem{c5} E. H. Miller, ``A note on reflector arrays (Periodical styleÑAccepted for publication),'' IEEE Trans. Antennas Propagat., to be publised.
\bibitem{c6} J. Wang, ``Fundamentals of erbium-doped fiber amplifiers arrays (Periodical styleÑSubmitted for publication),'' IEEE J. Quantum Electron., submitted for publication.
\bibitem{c7} C. J. Kaufman, Rocky Mountain Research Lab., Boulder, CO, private communication, May 1995.
\bibitem{c8} Y. Yorozu, M. Hirano, K. Oka, and Y. Tagawa, ``Electron spectroscopy studies on magneto-optical media and plastic substrate interfaces(Translation Journals style),'' IEEE Transl. J. Magn.Jpn., vol. 2, Aug. 1987, pp. 740--741 [Dig. 9th Annu. Conf. Magnetics Japan, 1982, p. 301].
\bibitem{c9} M. Young, The Techincal Writers Handbook.  Mill Valley, CA: University Science, 1989.
\bibitem{c10} J. U. Duncombe, ``Infrared navigationÑPart I: An assessment of feasibility (Periodical style),'' IEEE Trans. Electron Devices, vol. ED-11, pp. 34--39, Jan. 1959.
\bibitem{c11} S. Chen, B. Mulgrew, and P. M. Grant, ``A clustering technique for digital communications channel equalization using radial basis function networks,'' IEEE Trans. Neural Networks, vol. 4, pp. 570--578, July 1993.
\bibitem{c12} R. W. Lucky, ``Automatic equalization for digital communication,'' Bell Syst. Tech. J., vol. 44, no. 4, pp. 547--588, Apr. 1965.
\bibitem{c13} S. P. Bingulac, ``On the compatibility of adaptive controllers (Published Conference Proceedings style),'' in Proc. 4th Annu. Allerton Conf. Circuits and Systems Theory, New York, 1994, pp. 8--16.
\bibitem{c14} G. R. Faulhaber, ``Design of service systems with priority reservation,'' in Conf. Rec. 1995 IEEE Int. Conf. Communications, pp. 3--8.
\bibitem{c15} W. D. Doyle, ``Magnetization reversal in films with biaxial anisotropy,'' in 1987 Proc. INTERMAG Conf., pp. 2.2-1--2.2-6.
\bibitem{c16} G. W. Juette and L. E. Zeffanella, ``Radio noise currents n short sections on bundle conductors (Presented Conference Paper style),'' presented at the IEEE Summer power Meeting, Dallas, TX, June 22--27, 1990, Paper 90 SM 690-0 PWRS.
\bibitem{c17} J. G. Kreifeldt, ``An analysis of surface-detected EMG as an amplitude-modulated noise,'' presented at the 1989 Int. Conf. Medicine and Biological Engineering, Chicago, IL.
\bibitem{c18} J. Williams, ``Narrow-band analyzer (Thesis or Dissertation style),'' Ph.D. dissertation, Dept. Elect. Eng., Harvard Univ., Cambridge, MA, 1993. 
\bibitem{c19} N. Kawasaki, ``Parametric study of thermal and chemical nonequilibrium nozzle flow,'' M.S. thesis, Dept. Electron. Eng., Osaka Univ., Osaka, Japan, 1993.
\bibitem{c20} J. P. Wilkinson, ``Nonlinear resonant circuit devices (Patent style),'' U.S. Patent 3 624 12, July 16, 1990. 






\end{thebibliography}




\end{document}
