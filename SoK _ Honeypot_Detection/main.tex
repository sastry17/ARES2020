%%%%%%%%%%%%%%%%%%%%%%%%%%%%%%%%%%%%%%%%%%%%%%%%%%%%%%%%%%%%%%%%%%%%%%%%%%%%%%%%
%2345678901234567890123456789012345678901234567890123456789012345678901234567890
%        1         2         3         4         5         6         7         8

\documentclass[letterpaper, 10 pt, conference]{ieeeconf}  % Comment this line out
                                                          % if you need a4paper
%\documentclass[a4paper, 10pt, conference]{ieeeconf}      % Use this line for a4
                                                          % paper

\IEEEoverridecommandlockouts                              % This command is only
                                                          % needed if you want to
                                                          % use the \thanks command
\overrideIEEEmargins
% See the \addtolength command later in the file to balance the column lengths
% on the last page of the document

\usepackage[utf8]{inputenc}
\usepackage[T1]{fontenc}
\usepackage{float}
\usepackage{pgfplots}
\pgfplotsset{width=8cm,compat=1.9}
\usepackage{caption}
\usepackage{graphicx}
\usepackage{ragged2e}
\usepackage{subfiles}
\graphicspath{ {./Figures/} }

% The following packages can be found on http:\\www.ctan.org
%\usepackage{graphics} % for pdf, bitmapped graphics files
%\usepackage{epsfig} % for postscript graphics files
%\usepackage{mathptmx} % assumes new font selection scheme installed
%\usepackage{mathptmx} % assumes new font selection scheme installed
%\usepackage{amsmath} % assumes amsmath package installed
%\usepackage{amssymb}  % assumes amsmath package installed

\title{\LARGE \bf
SoK: Classifying Honeypot Fingerprinting Techniques
}

%%%% For Double Blind Version
%\author{Shreyas Srinivasa$^{1}$, Emmanouil Vasilomanolakis$^{2}$, Jens Myrup Pedersen$^{3}$% <-this % stops a space
%\thanks{*This work was not supported by any organization}% <-this % stops a space
%\thanks{$^{1}$S. Srinivasa - PhD Fellow at Department of Electronic Systems,
%        Aalborg University, Copenhagen, Denmark
%        {\tt\small shsr@es.aau.dk}}%
%\thanks{$^{2,3}$E. Vasilomanolakis, J. Pedersen - Asst. Professor at                          Department of Electronic Systems, Aalborg University,                         Copenhagen, Denmark
%        {\tt\small emv@es.aau.dk, jens@es.aau.dk}}%
%}


\begin{document}



\maketitle
\thispagestyle{empty}
\pagestyle{empty}


%%%%%%%%%%%%%%%%%%%%%%%%%%%%%%%%%%%%%%%%%%%%%%%%%%%%%%%%%%%%%%%%%%%%%%%%%%%%%%%%
\begin{abstract}
In defensive security, Honeypots play an important role to detect attacks by simulating the services of the target system. Over the years, many honeypots have been proposed, developed and deployed over the internet simulating essential protocols and services like HTTP, SSH, Telnet, SMB, SMTP, FTP and also industrial protocols like MODBUS and BARCnet. Survey statistics provided by F-Secure (rfr: F-secure white paper), show that the attack landscape caught by honeypots hosted by F-Secure, have tripled from 2018 with major targeted protocols being Telnet, SMB, SSH and  MSSQL. The rise in attacks can be in view of increase in smart IoT devices and also the accessibility of Cloud infrastructure. Further, resource constraints on IoT devices lead to poor security diligence to achieve performance and availability. Honeypots have been a reliable choice for proactive defense by  network administrators to detect and analyze attacks. Open-source low-interaction honeypots like Kippo, Cowrie, Glastopf, Dionaea and Conpot are the most widely deployed Honeypots (rfr: bitter harvest). As effective as  honeypots can be leveraged to detect attacks, they are also vulnerable to be exposed as decoys thereby reducing their productivity. The identity of honeypots and its ability to remain undetected is a valuable parameter towards its purpose. Studies have shown that these honeypots are easily detectable and can be fingerprint by various mechanisms. Probing, response analysis and  location/domain filtering are some methods through which the identity of honeypots can be revealed. In addition, machine learning based methods have inferred to increase the predictability and reduce false positives in detection. This paper provides an overview and a new basis for classification of honeypot fingerprinting mechanisms . 
\end{abstract}
%%%%%%%%%%%%%%%%%%%%%%%%%%%%%%%%%%%%%%%%%%%%%%%%%%%%%%%%%%%%%%%%%%%%%%%%%%%%%%%%

\section{INTRODUCTION}
\subfile{sections/1.introduction}

\section{Background}
\subfile{sections/2.background}

\section{Honeypot Fingerprinting Techniques}
\subfile{sections/3.HFT}

\section{Classifying Honeypot Fingerprinting Methods}
\subfile{sections/4.CHFM}   

\section{Evaluation of Honeypots based on Classification}
\subfile{sections/5.evaluation} 
 
\section {Countermeasures}
\subfile{sections/6.countermeasures}
 
\section {Related Work}
\subfile{sections/7.relatedwork} 
 
\section{CONCLUSIONS}
\subfile{sections/8.conclusions}

\addtolength{\textheight}{-12cm}   % This command serves to balance the column lengths
                                  % on the last page of the document manually. It shortens
                                  % the textheight of the last page by a suitable amount.
                                  % This command does not take effect until the next page
                                  % so it should come on the page before the last. Make
                                  % sure that you do not shorten the textheight too much.

%%%%%%%%%%%%%%%%%%%%%%%%%%%%%%%%%%%%%%%%%%%%%%%%%%%%%%%%%%%%%%%%%%%%%%%%%%%%%%%%
%%%%%%%%%%%%%%%%%%%%%%%%%%%%%%%%%%%%%%%%%%%%%%%%%%%%%%%%%%%%%%%%%%%%%%%%%%%%%%%%
%%%%%%%%%%%%%%%%%%%%%%%%%%%%%%%%%%%%%%%%%%%%%%%%%%%%%%%%%%%%%%%%%%%%%%%%%%%%%%%%
\section*{ACKNOWLEDGMENT}
%%%%%%%%%%%%%%%%%%%%%%%%%%%%%%%%%%%%%%%%%%%%%%%%%%%%%%%%%%%%%%%%%%%%%%%%%%%%%%%%
\bibliographystyle{plain}

\bibliography{references}

\end{document}
